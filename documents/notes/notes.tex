\documentclass{article}

\usepackage[ngerman]{babel}
\usepackage[utf8]{inputenc}
\usepackage[T1]{fontenc}
\usepackage{hyperref}
\usepackage{csquotes}

\usepackage[
    backend=biber,
    style=apa,
    sortlocale=de_DE,
    natbib=true,
    url=false,
    doi=false,
    sortcites=true,
    sorting=nyt,
    isbn=false,
    hyperref=true,
    backref=false,
    giveninits=false,
    eprint=false]{biblatex}
\addbibresource{../references/bibliography.bib}

\title{Notizen zum Projekt Data Ethics}
\author{Name des Autors}
\date{\today}

\begin{document}
\maketitle


\abstract{
    Dieses Dokument ist eine Sammlung von Notizen zu dem Projekt. Die Struktur innerhalb des
    Projektes ist gleich ausgelegt wie in der Hauptarbeit, somit kann hier einfach geschrieben
    werden, und die Teile die man verwenden möchte, kann man direkt in die Hauptdatei ziehen.
}

\tableofcontents

\section{Künstliche Intelligenz}
\label{sec:ai}

Die Künstliche Intelligenz ist ein Versuch menschliches Lernen und Intelligenz auf einen Computer zu übertragen und somit ihm Intelligenz zu verleihen.Algorithmus, Deep Learning und Machine Learning sind wichtige Begriffe im Bereich der Künstlichen Intelligenz. 

Das Verhalten der Technologien werden generell mit Mitteln der Informatik und Mathematik simuliert. Die Maschinen werden für bestimmte Aufgaben trainiert, indem sie mit grossen Datenmengen gefüttert werden und das Muster erkennen. Die KI verwendet künstliche neuronale Netze, um zu funktionieren. Dabei wird die Funktionsweise des Gehirns nachgeahmt. Der Unterschied, das Gehirn kann viel mehr leisten als alle Computer. "Computer stellen diese Informationsverarbeitung des Gehirns durch künstliche neuronale Netze nach.

Ein Algorithmus kann als Anleitung betrachtet werden, das einem Computer sagt, was er machen muss, um eine Aufgabe oder Problem zu lösen. Also muss der Computer wissen, was wir von ihm wollen, bevor er die Aufgabe erledigen kann. Dies wird von IT-Spezialisten programmiert, die dem Computer erklären, was sie von ihm erwünschen und Anweisungen geben, was das System machen soll

"Deep Learning"(deutsch: mehrschichtiges Lernen oder tiefgehendes Lernen)ist eine Kategorie des maschinellen Lernens. Im Prozess "Deep Learning"wird dem Computer beigebracht, wie man komplexe Muster aus grossen Datenmengen verstehen kann, indem Schichten von künstlichen neuronalen Netzwerken verwendet werden. Die Koniguration der Neuronen im Netzwerk basiert auf das Training mit grossen Datenmengen.

Ein weiterer wichtiger Begriff im Bereich der Künstlicher Intelligenz ist das "Machine Learning"(deutsch: Maschinelles Lernen), die Begabung des Computers aus Erfahrungen zu lernen und Erkenntnisse aus Daten zu gewinnen, ohne dafür extra programmiert zu sein.

Die KI ist nicht mehr wegzudenken, da sie bereits heute in vielen Bereichen unseres Alltags präsent ist. Wer nur an Chat Gpt denkt, liegt hier komplett falsch. Oft kommen wir in Berührung mit der KI, ohne es wirklich wahrzunehmen. Ob Schachspiel gegen einen Computer, Besuch eines Online Shops, Interaktion mit Sprachassistenten oder Verwendung eines Übersetzungsdienstes. In all diesen Situationen kommen wir in Kontakt mit der Künstlichen Intelligenz. Von einer Schwachen KI spricht man, wenn die KI auf den Bereich beschränkt ist, indem sie eingesetzt wird, also nur in einem spezifschen Bereich ziemlich gut ist.

Die KI, der grosse technologische Fortschritt der heutigen Zeit, ihre Entwicklung schreitet rasch
voran und die Einsatzmöglichkeiten nehmen zu. Neben den vielen Vorteilen, besteht auch eine
Gefahr, dass die KI eines Tages schlauer wird als Menschen. Steht also wirklich das Verderben
der Menschheit durch die KI vor? Kann die KI aber tatsächlich Arbeitskräfte ersetzen? Die KI,
ein Segen oder doch ein Fluch? Die KI wird in der Zukunft den Arbeitsalltag vieler Menschen
ändern. Studien zeigen, dass Berufe wie z.B Mathematiker oder Programmierer am meisten be-
droht sind. Das muss jedoch nicht unbedingt heissen, dass viele Arbeitskräfte ihren Beruf nicht
mehr ausführen können, sondern dass sie von lästigen Aufgaben entlastet werden und mehr Zeit
haben für Tätigkeiten, bei denen man nicht auf Menschen verzichten kann. Die KI kann zum Beispiel viele Bereiche der Medizin revolutionieren. Die Themen Rassismus und Sexismus kommen
im Bereich der KI leider nicht selten vor, da KI- Systeme wie z.B Chat Bots aus Daten lernen,
welche häufig nicht neutral sind, wiederspiegelt das System dies und kann diskriminierende Entscheidungen treffen. In den letzten Wochen und Monaten haben Nachrichten über Deep Fakes
durch AI zugenommen. Die Künstliche Intelligenz erlaubt es uns Fotos, Videos als auch Audiodateien zu manipulieren. Viele Stars und Politiker wurdern bereits Opfer von den sogenannten
Deep Fakes. All diese Szenarien zeigen, dass wir sehr verantwortungsbewusst mit der KI umgehen müssen und ihre Einsatzmöglichkeiten wie z.B das Erstellen von Deep Fakes beschränken
müssen. Jedoch will ich noch betonen, dass sich hinter der KI einige Gefahren verbergen, aber
auch sehr viele positive Aspekte. In den sozialen Medien wird die Gefahr der KI übertrieben
dargestellt, was eher einer unrealistischen Fiktion entspricht



\printbibliography

\end{document}

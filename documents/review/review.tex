\documentclass{article}

\usepackage[ngerman]{babel}
\usepackage[utf8]{inputenc}
\usepackage[T1]{fontenc}
\usepackage{hyperref}
\usepackage{csquotes}

\usepackage[
    backend=biber,
    style=apa,
    sortlocale=de_DE,
    natbib=true,
    url=false,
    doi=false,
    sortcites=true,
    sorting=nyt,
    isbn=false,
    hyperref=true,
    backref=false,
    giveninits=false,
    eprint=false]{biblatex}
\addbibresource{../references/bibliography.bib}

\title{Review des Papers "Ethischer Umgang mit Daten im Zusammenhang mit der KI" von Max Brügger}
\author{Dishan Selvanantham}
\date{\today}

\begin{document}
\maketitle

\abstract{
    Dies ist ein Review der Arbeit zum Thema "Ethischer Umgang mit Daten im Zusammenhang mit der KI" von <Max Brügger>.
}

\section{Positive Aspekte}

Die Arbeit bietet eine umfassende Einführung in das Thema "Künstliche Intelligenz im medizinischen Bereich. Von Max Brügger wurden verschiedene Anwendungsgebiete von KI in der Medizin wie z.B Diagonose, persönliche Medizin und Bildgebung behandelt.
Das ausgewählte Thema wird exakt beschrieben und das Expertenwissen kann erkannt werden. Viele Funktionen von Latex wie z.B das Einfügen von Listen und Bildern wurden genutzt. Das Inhaltsverzeichnis hilft dem Leser sich zu orientieren und anhand des Literaturverzeichnisses und des Textes kann man erkennen, dass gute Quellen ausgewählt wurden. Der Leser kann aus der Arbeit vieles Neues lernen, auch wenn er sich mit dem Thema nicht auskennt.


\section{Negative Aspekte}


Einige Abschnitte wie z.B die Einleitung sind zu überflüssig und könnten gekürzt werden. In der Arbeit fehlt mir ein bisschen die Struktur, was die Orientierung des Lesers beeinträchtigt. die Texte hätten in mehrere Kapitel mit Überschriften gegliedert sein können.
Es gibt zwar ein Bild am Anfang der Arbeit, doch leider fehlt dies in den Texten und somit ist es etwas schwerer für den Leser das Ganze vorzustellen. 

\section{Korrektur}

Ich konnte in der Arbeit keine falschen Aussagen finden. Die dargestellten Fakten und ethischen Überlegungen entsprechen der aktuellen KI- Forschung.
Im Text gibt es einige Rechtschreibefehler wie z.B die Kommasetzung und einige Wörter fehlen oder wurden im Text falsch geschrieben.

\section{Vorschläge}


\printbibliography

\end{document}

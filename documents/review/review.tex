\documentclass{article}

\usepackage[ngerman]{babel}
\usepackage[utf8]{inputenc}
\usepackage[T1]{fontenc}
\usepackage{hyperref}
\usepackage{csquotes}

\usepackage[
    backend=biber,
    style=apa,
    sortlocale=de_DE,
    natbib=true,
    url=false,
    doi=false,
    sortcites=true,
    sorting=nyt,
    isbn=false,
    hyperref=true,
    backref=false,
    giveninits=false,
    eprint=false]{biblatex}
\addbibresource{../references/bibliography.bib}

\title{Review des Papers "Ethischer Umgang mit Daten im Zusammenhang mit der KI" von Max Brügger}
\author{Dishan Selvanantham}
\date{\today}

\begin{document}
\maketitle

\abstract{
    Dies ist ein Review der Arbeit zum Thema "Ethischer Umgang mit Daten im Zusammenhang mit der KI" von <Max Brügger>.
}

\section{Positiven Aspekte}

Die Arbeit bietet eine umfassende Einführung in das Thema "Künstliche Intelligenz im medizinischen Bereich. Von Max Brügger wurden verschiedene Anwendungsgebiete von KI in der Medizin wie z.B Diagonose, persönliche Medizin und Bildgebung behandelt.
Das ausgewählte Thema wird exakt beschrieben und das Expertenwissen kann erkannt werden. 


\section{Negativen Aspekte}

Einige Abschnitte sind zu überflüssig und könnten gekürzt werden. In der Arbeit fehlt mir ein bisschen die Struktur, die Texte hätten in mehrere Kapitel mit Überschriften gegliedert sein können.
Es gibt zwar ein Bild am Anfang der Arbeit, doch leider fehlt dies in den Texten und somit ist es etwas schwerer für den Leser das Ganze vorzustellen.
\printbibliography

\end{document}

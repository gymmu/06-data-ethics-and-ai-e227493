\section{KI-Ethik}
\label{sec:data}

KI- Ethik ist ein Teilgebiet, das sich mit ethischen Fragen in Zusammenhang mit der Künstlichen
Intelligenz befasst. Die Ethik in der KI soll sicherstellen, dass die Künstliche Intelligenz nur zum
Wohle der Gesellschaft verwendet wird und keine Schäden verusacht.

Die Ethik im Bereich der KI bezieht sich auf verschiedene Bereiche, darunter die Rolle der KI in
der Gesellschaft und die ethischen Werte. Ein wichtiger Aspekt der KI ist das ethische Verhalten
der Maschinen, der Einsatz muss den Grundsätzen der Rechtmässigkeit, Transparenz, Sicherheit
und Verantwortlichkeit entsprechen. Die KI darf nie diskriminieren oder Vorurteile verstärken.
Die Gestaltung der KI muss der Würde des Menschen und den Grundrechten entsprechen. Bevor
dem Einsatz der KI müssen die Risiken bewertet werden, um potenzielle negative Auswirkungen
zu erkennen und zu minimieren. Die Einhaltung ethischer Grundsätze ist im Bereich der KI von
entscheidender Bedeutung, damit KI-Systeme verantwortungsbewusst eingesetzt werden.


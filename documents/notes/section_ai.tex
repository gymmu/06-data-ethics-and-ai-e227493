\section{Künstliche Intelligenz}
\label{sec:ai}

Die Künstliche Intelligenz ist ein Versuch menschliches Lernen und Intelligenz auf einen Computer zu übertragen und somit ihm Intelligenz zu verleihen.Algorithmus, Deep Learning und Machine Learning sind wichtige Begriffe im Bereich der Künstlichen Intelligenz. 

Das Verhalten der Technologien werden generell mit Mitteln der Informatik und Mathematik simuliert. Die Maschinen werden für bestimmte Aufgaben trainiert, indem sie mit grossen Datenmengen gefüttert werden und das Muster erkennen. Die KI verwendet künstliche neuronale Netze, um zu funktionieren. Dabei wird die Funktionsweise des Gehirns nachgeahmt. Der Unterschied, das Gehirn kann viel mehr leisten als alle Computer. "Computer stellen diese Informationsverarbeitung des Gehirns durch künstliche neuronale Netze nach.

Ein Algorithmus kann als Anleitung betrachtet werden, das einem Computer sagt, was er machen muss, um eine Aufgabe oder Problem zu lösen. Also muss der Computer wissen, was wir von ihm wollen, bevor er die Aufgabe erledigen kann. Dies wird von IT-Spezialisten programmiert, die dem Computer erklären, was sie von ihm erwünschen und Anweisungen geben, was das System machen soll

"Deep Learning"(deutsch: mehrschichtiges Lernen oder tiefgehendes Lernen)ist eine Kategorie des maschinellen Lernens. Im Prozess "Deep Learning"wird dem Computer beigebracht, wie man komplexe Muster aus grossen Datenmengen verstehen kann, indem Schichten von künstlichen neuronalen Netzwerken verwendet werden. Die Koniguration der Neuronen im Netzwerk basiert auf das Training mit grossen Datenmengen.

Ein weiterer wichtiger Begriff im Bereich der Künstlicher Intelligenz ist das "Machine Learning"(deutsch: Maschinelles Lernen), die Begabung des Computers aus Erfahrungen zu lernen und Erkenntnisse aus Daten zu gewinnen, ohne dafür extra programmiert zu sein.

Die KI ist nicht mehr wegzudenken, da sie bereits heute in vielen Bereichen unseres Alltags präsent ist. Wer nur an Chat Gpt denkt, liegt hier komplett falsch. Oft kommen wir in Berührung mit der KI, ohne es wirklich wahrzunehmen. Ob Schachspiel gegen einen Computer, Besuch eines Online Shops, Interaktion mit Sprachassistenten oder Verwendung eines Übersetzungsdienstes. In all diesen Situationen kommen wir in Kontakt mit der Künstlichen Intelligenz. Von einer Schwachen KI spricht man, wenn die KI auf den Bereich beschränkt ist, indem sie eingesetzt wird, also nur in einem spezifschen Bereich ziemlich gut ist.



